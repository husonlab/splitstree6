\documentclass{article}
\usepackage{graphicx}
\usepackage{hyperref}
\usepackage{fullpage}
\usepackage{parskip} 

\title{SplitsTree Community Edition - Algorithms}
\author{Daniel H. Huson and Dave Bryant}
\date{March 2024}

\begin{document}

\maketitle


\section{List of algorithms sorted by input}


\subsection{Input Characters Block}

\subsubsection{Base Freq Distance}

The ``Base Freq Distance'' algorithm takes a Characters block as input and produces a Distances block as output. It calculates distances from differences in the base composition.

\subsubsection{Binary To Splits}

The ``Binary To Splits'' algorithm takes a Characters block as input and produces a Splits block as output. It converts binary characters directly into splits.

{\footnotesize\obeylines
\verb^MinSplitWeight =  <Double>^ - minimum split weight threshold
\verb^HighDimensionFilter =  <Boolean>^ - activate high-dimensional filter to avoid exponential graph size
\verb^AddAllTrivial =  <Boolean>^ - ensure all trival splits are present
}

See: D.H. Huson, R. Rupp and C. Scornavacca, Phylogenetic Networks, Cambridge, 2012.

\subsubsection{Characters Filter}

The ``Characters Filter'' algorithm takes a Characters block as input and produces a Characters block as output. It provides several ways of filtering characters.

{\footnotesize\obeylines
\verb^ExcludeGapSites =  <Boolean>^ - exclude all sites that contain a gap
\verb^ExcludeParsimonyUninformativeSites =  <Boolean>^ - exclude all sites that are parsimony uninformative
\verb^ExcludeConstantSites =  <Boolean>^ - exclude all sites that are constant
\verb^ExcludeFirstCodonPosition =  <Boolean>^ - exclude first and then every third site
\verb^ExcludeSecondCodonPosition =  <Boolean>^ - exclude second and then every third site
\verb^ExcludeThirdCodonPosition =  <Boolean>^ - exclude third and then every third site
}

\subsubsection{Codominant}

The ``Codominant'' algorithm takes a Characters block as input and produces a Distances block as output. It computes the co-dominant genetic distance.

See:  Smouse PE, Peakall R. Spatial autocorrelation analysis of individual multiallele and multilocus genetic structure. Heredity, 82, 561-573, 1999.

\subsubsection{Dice}

The ``Dice'' algorithm takes a Characters block as input and produces a Distances block as output. It computes distances using the Dice coefficient distance.

See:  Dice, Lee R. (1945). Measures of the Amount of Ecologic Association Between Species. Ecology. 26 (3): 297�302.

\subsubsection{DNA To Splits}

The ``DNA To Splits'' algorithm takes a Characters block as input and produces a Splits block as output. It converts DNA characters directly into splits.

{\footnotesize\obeylines
\verb^Method =  {MajorityState | RYAlphabet}^ - use either majority-state-vs-others or RY alphabet
\verb^MinSplitWeight =  <Double>^ - minimum split weight threshold
\verb^HighDimensionFilter =  <Boolean>^ - activate high-dimensional filter to avoid exponential graph size
}

See: D.H. Huson, R. Rupp and C. Scornavacca, Phylogenetic Networks, Cambridge, 2012.

\subsubsection{Estimate Invariable Sites}

The ``Estimate Invariable Sites'' algorithm takes a Characters block as input and produces a Report block as output. It estimates the proportion of invariant sites using capture-recapture.

See:  MA Steel, DH Huson, and PJ Lockhart. Invariable site models and their use in phylogeny reconstruction. Sys. Biol. 49(2):225-232, 2000

\subsubsection{External Program}

The ``External Program'' algorithm takes a Characters block as input and produces a Trees block as output. It runs an external program.

{\footnotesize\obeylines
\verb^Name =  <String>^ - specify a name for this calculation
\verb^ProgramCall =  <String>^ - specification of external program: replace 'path-to-program' by path to program and
\verb^use '%i' and '%o' as place^ - holders for the program's input and output files
\verb^CharactersFormat =  {Phylip | Nexus | FastA}^ - specify the format to write out the current data in
}

\subsubsection{F81}

The ``F81'' algorithm takes a Characters block as input and produces a Distances block as output. It calculates distances under the Felsenstein81 model.

{\footnotesize\obeylines
\verb^BaseFrequencies =  <doubleArray>^ - base frequencies (in order ACGT/U)
\verb^SetBaseFrequencies =  {fromChars | defaultValues}^ - set base frequencies to default values, or to estimations from characters (using Capture-recapture for invariable sites)
\verb^PropInvariableSites =  <Double>^ - proportion of invariable sites
\verb^SetSiteVarParams =  {fromChars | defaultValues}^ - set site variation parameters to default values, or to estimations from characters
\verb^UseML_Distances =  <Boolean>^ - use maximum likelihood estimation of distances (rather than exact distances)
}

See:  Felsenstein J (1981). Evolutionary trees from DNA sequences: a maximum likelihood approach. Journal of Molecular Evolution. 17 (6): 368�376.

\subsubsection{F84}

The ``F84'' algorithm takes a Characters block as input and produces a Distances block as output. It calculates distances under the Felsenstein84 model.

{\footnotesize\obeylines
\verb^BaseFrequencies =  <doubleArray>^ - base frequencies (in order ACGT/U)
\verb^SetBaseFrequencies =  {fromChars | defaultValues}^ - set base frequencies to default values, or to estimations from characters (using Capture-recapture for invariable sites)
\verb^PropInvariableSites =  <Double>^ - proportion of invariable sites
\verb^SetSiteVarParams =  {fromChars | defaultValues}^ - set site variation parameters to default values, or to estimations from characters
\verb^UseML_Distances =  <Boolean>^ - use maximum likelihood estimation of distances (rather than exact distances)
}

See:  Felsenstein J, Churchill GA (1996). A Hidden Markov Model approach to variation among sites in rate of evolution, and the branching order in hominoidea. Molecular Biology and Evolution. 13 (1): 93�104.

\subsubsection{GTR}

The ``GTR'' algorithm takes a Characters block as input and produces a Distances block as output. It calculates distances under the general time-reversible model.

{\footnotesize\obeylines
\verb^PropInvariableSites =  <Double>^ - proportion of invariable sites
\verb^SetSiteVarParams =  {fromChars | defaultValues}^ - set site variation parameters to default values, or to estimations from characters
\verb^RateMatrix =  <doubleSquareMatrix>^ - rate matrix for GTR (in order ACGT/U)
\verb^UseML_Distances =  <Boolean>^ - use maximum likelihood estimation of distances (rather than exact distances)
}

See:  Tavar� S (1986). Some Probabilistic and Statistical Problems in the Analysis of DNA Sequences. Lectures on Mathematics in the Life Sciences. 17: 57�86

\subsubsection{Gene Content Distance}

The ``Gene Content Distance'' algorithm takes a Characters block as input and produces a Distances block as output. It computes distances based on the presence/absence of genes.

{\footnotesize\obeylines
\verb^Method =  {MLE | SharedGenes}^ - choose Maximum likelihood distance estimation (Huson and Steel 2004, eq. 4), or shared genes distance (Snel et al, 1997)
}

See:  D.H. Huson  and  M. Steel. Phylogenetic  trees  based  on  gene  content. Bioinformatics, 20(13):2044�9, 2004.

\subsubsection{HKY85}

The ``HKY85'' algorithm takes a Characters block as input and produces a Distances block as output. It calculates distances under the Hasegawa-Kishino-Yano model.

{\footnotesize\obeylines
\verb^TsTvRatio =  <Double>^ - ratio of transitions vs transversions
\verb^BaseFrequencies =  <doubleArray>^ - base frequencies (in order ACGT/U)
\verb^SetBaseFrequencies =  {fromChars | defaultValues}^ - set base frequencies to default values, or to estimations from characters (using Capture-recapture for invariable sites)
\verb^PropInvariableSites =  <Double>^ - proportion of invariable sites
\verb^SetSiteVarParams =  {fromChars | defaultValues}^ - set site variation parameters to default values, or to estimations from characters
}

See:  Hasegawa M, Kishino H, Yano T. Dating of human-ape splitting by a molecular clock of mitochondrial DNA. Journal of Molecular Evolution. 22 (2): 160�174. PMID 3934395. doi:10.1007/BF02101694, 1985.

\subsubsection{Hamming Distances}

The ``Hamming Distances'' algorithm takes a Characters block as input and produces a Distances block as output. It computes distances based on the number of character-state differences.

{\footnotesize\obeylines
\verb^Normalize =  <Boolean>^ - normalize distances
}

See:  Hamming, Richard W. Error detecting and error correcting codes. Bell System Technical Journal. 29 (2): 147�160. MR 0035935, 1950.

\subsubsection{Hamming Distances Ambig States}

The ``Hamming Distances Ambig States'' algorithm takes a Characters block as input and produces a Distances block as output. It computes distances based on the number of character-state differences.

{\footnotesize\obeylines
\verb^Normalize =  <Boolean>^ - normalize distances
\verb^HandleAmbiguousStates =  {Ignore | AverageStates | MatchStates}^ - choose how to handle ambiguous states
}

See:  Hamming, Richard W. Error detecting and error correcting codes. Bell System Technical Journal. 29 (2): 147�160. MR 0035935, 1950.

\subsubsection{Jaccard}

The ``Jaccard'' algorithm takes a Characters block as input and produces a Distances block as output. It computes distances based on the Jaccard index.

See:  Jaccard, Paul (1901). �tude comparative de la distribution florale dans une portion des Alpes et des Jura, Bulletin de la Soci�t� Vaudoise des Sciences Naturelles, 37: 547�579.

\subsubsection{Jukes Cantor}

The ``Jukes Cantor'' algorithm takes a Characters block as input and produces a Distances block as output. It calculates distances under the Jukes-Cantor model.

{\footnotesize\obeylines
\verb^PropInvariableSites =  <Double>^ - proportion of invariable sites
\verb^SetSiteVarParams =  {fromChars | defaultValues}^ - set site variation parameters to default values, or to estimations from characters
\verb^UseML_Distances =  <Boolean>^ - use maximum likelihood estimation of distances (rather than exact distances)
}

See:  Jukes TH \& Cantor CR (1969). Evolution of Protein Molecules. New York: Academic Press. pp. 21�132

\subsubsection{K2P}

The ``K2P'' algorithm takes a Characters block as input and produces a Distances block as output. It calculates distances under the Kimura-2P model.

{\footnotesize\obeylines
\verb^TsTvRatio =  <Double>^ - ratio of transitions vs transversions
\verb^Gamma =  <Double>^ - alpha value for the Gamma distribution
\verb^PropInvariableSites =  <Double>^ - proportion of invariable sites
\verb^SetSiteVarParams =  {fromChars | defaultValues}^ - set site variation parameters to default values, or to estimations from characters
\verb^UseML_Distances =  <Boolean>^ - use maximum likelihood estimation of distances (rather than exact distances)
}

See:  Kimura M (1980). A simple method for estimating evolutionary rates of base substitutions through comparative studies of nucleotide sequences. Journal of Molecular Evolution. 16 (2): 111�120.

\subsubsection{K3ST}

The ``K3ST'' algorithm takes a Characters block as input and produces a Distances block as output. It calculates distances under the Kimura-3P model.

{\footnotesize\obeylines
\verb^TsTvRatio =  <Double>^ - ratio of transitions vs transversions
\verb^ACvATRatio =  <Double>^ - optionACvATRatio
\verb^Gamma =  <Double>^ - alpha value for the Gamma distribution
\verb^PropInvariableSites =  <Double>^ - proportion of invariable sites
\verb^SetSiteVarParams =  {fromChars | defaultValues}^ - set site variation parameters to default values, or to estimations from characters
\verb^UseML_Distances =  <Boolean>^ - use maximum likelihood estimation of distances (rather than exact distances)
}

See:  M. Kimura, Estimation of evolutionary sequences between homologous nucleotide sequences, Proc. Natl. Acad. Sci. USA 78 (1981) 454�45

\subsubsection{Log Det}

The ``Log Det'' algorithm takes a Characters block as input and produces a Distances block as output. It computes distances using the Log-Det method.

{\footnotesize\obeylines
\verb^PropInvariableSites =  <Double>^ - proportion of invariable sites
\verb^FudgeFactor =  <Boolean>^ - input missing matrix entries using LDDist method
\verb^FillZeros =  <Boolean>^ - replace zeros with small numbers in rows/columns with values
}

See:  M.A. Steel. Recovering a tree from the leaf colorations it generates under a Markov model. Appl. Math. Lett., 7(2):19�24, 1994.

\subsubsection{Median Joining}

The ``Median Joining'' algorithm takes a Characters block as input and produces a Network block as output. It computes a haplotype network using the median-joining method.

{\footnotesize\obeylines
\verb^Epsilon =  <Integer>^ - balances accuracy (smaller value) and efficiency (larger value)
}

See: H. -J. Bandelt, P. Forster, and A. R�hl. Median-joining networks for inferring intraspecific phylogenies. Molecular Biology and Evolution, 16:37�48, 1999.

\subsubsection{Nei Li Restriction Distance}

The ``Nei Li Restriction Distance'' algorithm takes a Characters block as input and produces a Distances block as output. It calculates distances for restriction data.

See: M Nei and W H Li. Mathematical model for studying genetic variation in terms of restriction endonucleases, PNAS 79(1):5269-5273, 1979.

\subsubsection{Nei Miller}

The ``Nei Miller'' algorithm takes a Characters block as input and produces a Distances block as output. It estimates the average number of nucleotide substitutions from restriction data.

See:  M. Nei and J.C. Miller. A simple method for estimating average number of nucleotide substitutions within and between populations from restriction data. Genetics, 125:873�879, 1990.

\subsubsection{Parsimony Splits}

The ``Parsimony Splits'' algorithm takes a Characters block as input and produces a Splits block as output. It computes weakly-compatible splits directly from DNA characters.

See:  H.-J.Bandelt and A.W.M.Dress. A canonical decomposition theory for metrics on a finite set. Advances in Mathematics, 92:47�105, 1992.

\subsubsection{Phi Test}

The ``Phi Test'' algorithm takes a Characters block as input and produces a Report block as output. It performs a statistical test for detecting the presence of recombination.

See:  Bruen TC, Philippe H, Bryant D. A simple and robust statistical test for detecting the presence of recombination. Genetics 17(4):2665-81, 2006

\subsubsection{Protein ML Dist}

The ``Protein ML Dist'' algorithm takes a Characters block as input and produces a Distances block as output. It computes distances for proteins using maximum-likelihood estimation.

{\footnotesize\obeylines
\verb^Model =  {cpREV45 | Dayhoff | JTT | mtMAM | mtREV24 | pmb | Rhodopsin | WAG}^ - choose an amino acid substitution model
\verb^PropInvariableSites =  <Double>^ - proportion of invariable sites
\verb^Gamma =  <Double>^ - alpha parameter for gamma distribution. Negative gamma = Equal rates
}

See:  D.L. Swofford, G.J. Olsen, P.J. Waddell, and  D.M. Hillis. Chapter  11:  Phylogenetic inference. In D. M. Hillis, C. Moritz, and B. K. Mable, editors, Molecular Systematics, pages 407�514. Sinauer Associates, Inc., 2nd edition, 1996.

\subsubsection{Uncorrected P}

The ``Uncorrected P'' algorithm takes a Characters block as input and produces a Distances block as output. It computes distances based on the number of character-state differences.

{\footnotesize\obeylines
\verb^Normalize =  <Boolean>^ - normalize distances
}

See:  Hamming, Richard W. Error detecting and error correcting codes. Bell System Technical Journal. 29 (2): 147�160. MR 0035935, 1950.

\subsubsection{Upholt}

The ``Upholt'' algorithm takes a Characters block as input and produces a Distances block as output. It calculates distances for restriction data.

See:  Upholt WB. Estimation of DNA sequence divergence from comparison of restriction endonuclease digests. Nucleic Acids Res. 1977;4(5):1257-65.

\subsection{Input Distances Block}

\subsubsection{Bio NJ}

The ``Bio NJ'' algorithm takes a Distances block as input and produces a Trees block as output. It computes an unrooted phylogenetic tree using the Bio-NJ method.

See:  O. Gascuel, BIONJ: an improved version of the NJ algorithm based on a simple model of sequence data. Molecular Biology and Evolution. 1997 14:685-695.

\subsubsection{Buneman Tree}

The ``Buneman Tree'' algorithm takes a Distances block as input and produces a Splits block as output. It computes a set of compatible splits using the Buneman tree method.

See:  H.-J. Bandelt and A.W.M.Dress. A canonical decomposition theory for metrics on a finite set. Advances in Mathematics, 92:47�105, 1992.

\subsubsection{Delta Score}

The ``Delta Score'' algorithm takes a Distances block as input and produces a Report block as output. It calculates the delta score.

See: B. R. Holland, K. T. Huber, A. Dress, V. Moulton, Delta Plots: A tool for analyzing phylogenetic distance data, Molecular Biology and Evolution, 19(12):2051�2059, 2002.

\subsubsection{Min Spanning Network}

The ``Min Spanning Network'' algorithm takes a Distances block as input and produces a Network block as output. It computes a minimum spanning network.

{\footnotesize\obeylines
\verb^Epsilon =  <Double>^ - weighted genetic distance measure. Low: MedianJoining, High: full median network
\verb^MinSpanningTree =  <Boolean>^ - calculate minimum spanning tree
}

See:  Excoffier L, Smouse PE. Using allele frequencies and geographic subdivision to reconstruct gene trees within a species: molecular variance parsimony (1994) Genetics.136(1):343-59.

\subsubsection{Min Spanning Tree}

The ``Min Spanning Tree'' algorithm takes a Distances block as input and produces a Trees block as output. It computes a minimum spanning tree.

See:  Excoffier L, Smouse PE. Using allele frequencies and geographic subdivision to reconstruct gene trees within a species: molecular variance parsimony (1994) Genetics.136(1):343-59.

\subsubsection{Neighbor Joining}

The ``Neighbor Joining'' algorithm takes a Distances block as input and produces a Trees block as output. It computes an unrooted phylogenetic tree using the neighbor-joining method.

See:  N. Saitou and M. Nei. The Neighbor-Joining method: a new method for reconstructing phylogenetic trees. Molecular Biology and Evolution, 4:406-425, 1987.

\subsubsection{Neighbor Net}

The ``Neighbor Net'' algorithm takes a Distances block as input and produces a Splits block as output. It computes a set of cyclic splits using the neighbor-net method.

{\footnotesize\obeylines
\verb^InferenceAlgorithm =  {GradientProjection | ActiveSet | APGD | SplitsTree4}^ - the inference algorithm to be used
}

See:  D. Bryant and V. Moulton. Neighbor-net: An agglomerative method for the construction of phylogenetic networks. Molecular Biology and Evolution, 21(2):255� 265, 2004.;Bryant \& Huson 2023;D. Bryant and D.H. Huson, NeighborNet- improved algorithms and implementation. Front. Bioinform. 3, 2023

\subsubsection{P Co A}

The ``P Co A'' algorithm takes a Distances block as input and produces a Network block as output. It performs principal coordinates analysis.

{\footnotesize\obeylines
\verb^FirstCoordinate =  <Integer>^ - choose principal component for the x Axis
\verb^SecondCoordinate =  <Integer>^ - choose principal component for the y Axis
}

See:  Gower, J. C. (1966). Some distance properties of latent root and vector methods used in multivariate analysis. Biometrika, 53(3-4), 325-338.

\subsubsection{Split Decomposition}

The ``Split Decomposition'' algorithm takes a Distances block as input and produces a Splits block as output. It computes a set of weakly-compatible splits using the split-decomposition method.

See:  H.-J.Bandelt and A.W.M.Dress. A canonical decomposition theory for metrics on a finite set. Advances in Mathematics, 92:47�105, 1992.

\subsubsection{UPGMA}

The ``UPGMA'' algorithm takes a Distances block as input and produces a Trees block as output. It computes a rooted phylogenetic tree using the UPGMA method.

See:  R.R. Sokal and C.D.Michener. A statistical method for evaluating systematic relationships. University of Kansas Scientific Bulletin, 28:1409-1438, 1958.

\subsection{Input Splits Block}

\subsubsection{Bootstrap Splits}

The ``Bootstrap Splits'' algorithm takes a Splits block as input and produces a Splits block as output. It performs bootstrapping on splits.

{\footnotesize\obeylines
\verb^Replicates =  <Integer>^ - number of bootstrap replicates
\verb^MinPercent =  <Double>^ - minimum percentage support for a split to be included
\verb^ShowAllSplits =  <Boolean>^ - show all bootstrap splits, not just the original splits
\verb^RandomSeed =  <Integer>^ - if non-zero, is used as seed for random number generator
\verb^HighDimensionFilter =  <Boolean>^ - heuristically remove splits causing high-dimensional network
}

See: Felsenstein J. Confidence limits on phylogenies: an approach using the bootstrap. Evolution. 1985;39(4):783-791

\subsubsection{Dimension Filter}

The ``Dimension Filter'' algorithm takes a Splits block as input and produces a Splits block as output. It heuristically removes splits that lead to high-dimensional boxes in a split network.

{\footnotesize\obeylines
\verb^MaxDimension =  <Integer>^ - heuristically remove splits that create configurations of a higher dimension than this threshold
}

\subsubsection{Greedy Tree}

The ``Greedy Tree'' algorithm takes a Splits block as input and produces a Trees block as output. It produces a phylogenetic tree based on greedily selecting a compatible set of splits.

See: D.H. Huson, R. Rupp and C. Scornavacca, Phylogenetic Networks, Cambridge, 2012.

\subsubsection{Phylogenetic Diversity}

The ``Phylogenetic Diversity'' algorithm takes a Splits block as input and produces a Report block as output. It calculates the phylogenetic diversity for selected taxa.

See: Volkmann L, Martyn I, Moulton V, Spillner A, Mooers AO. Prioritizing populations for conservation using phylogenetic networks. PLoS ONE 9(2):e88945 (2014)

\subsubsection{Shapley Values}

The ``Shapley Values'' algorithm takes a Splits block as input and produces a Report block as output. It calculates Shapley values on splits.

See: Volkmann L, Martyn I, Moulton V, Spillner A, Mooers AO. Prioritizing populations for conservation using phylogenetic networks. PLoS ONE 9(2):e88945 (2014)

\subsubsection{Show Splits}

The ``Show Splits'' algorithm takes a Splits block as input and produces a View block as output. It provides interactive visualizations of split networks.

{\footnotesize\obeylines
\verb^View =  {SplitsNetwork}^ - the type of splits viewer to use
}

\subsubsection{Splits Filter}

The ``Splits Filter'' algorithm takes a Splits block as input and produces a Splits block as output. It provides several ways of filtering splits.

{\footnotesize\obeylines
\verb^WeightThreshold =  <Float>^ - set minimum split weight threshold
\verb^ConfidenceThreshold =  <Float>^ - set the minimum split confidence threshold
\verb^MaximumDimension =  <Integer>^ - set maximum dimension threshold (necessary to avoid computational overload)
\verb^FilterAlgorithm =  {None | GreedyCompatible | GreedyCircular | GreedyWeaklyCompatible | BlobTree}^ - set the filter algorithm
\verb^RecomputeCycle =  <Boolean>^ - recompute circular ordering
}

\subsubsection{Weights Slider}

The ``Weights Slider'' algorithm takes a Splits block as input and produces a Splits block as output. It allows one to interactively filter splits by their weight.

{\footnotesize\obeylines
\verb^WeightThreshold =  <Double>^ - set minimum split weight threshold
}

\subsection{Input Trees Block}

\subsubsection{ALTS External}

The ``ALTS External'' algorithm takes a Trees block as input and produces a Trees block as output. It runs an external implementation of the ALTS algorithm.

{\footnotesize\obeylines
\verb^ALTSExecutableFile =  <String>^ - download and compile ALTSNetwork program from https://github.com/LX-Zhang/AAST, then set this parameter to the executable.
Note that the program requires fully resolved trees as input and any unresolved trees will be ignored
}

See:  Louxin Zhang, Niloufar Niloufar Abhari, Caroline Colijn and Yufeng Wu3. A fast and scalable method for inferring phylogenetic networks from trees by aligning lineage taxon strings. Genome Res. 2023

\subsubsection{ALTS Network}

The ``ALTS Network'' algorithm takes a Trees block as input and produces a Trees block as output. It computes one or more rooted networks that contain all input trees using the M-ALTS algorithm.

{\footnotesize\obeylines
\verb^MutualRefinement =  <Boolean>^ - mutually refine trees during preprocessing
\verb^RemoveDuplicates =  <Boolean>^ - remove duplicate networks in output
\verb^Kernelization =  <Boolean>^ - perform kernelization during preprocessing
}

See:  Louxin Zhang, Niloufar Niloufar Abhari, Caroline Colijn and Yufeng Wu3. A fast and scalable method for inferring phylogenetic networks from trees by aligning lineage taxon strings. Genome Res. 2023;Zhang et al, 2024; Louxin Zhang, Banu Cetinkaya and Daniel H Huson. Hybrization networks from multiple trees, in preparation.

\subsubsection{Autumn Algorithm}

The ``Autumn Algorithm'' algorithm takes a Trees block as input and produces a Trees block as output. It computes all minimum hybridization networks using the Autumn algorithm

{\footnotesize\obeylines
\verb^FirstTree =  <Integer>^ - index of the first tree
\verb^SecondTree =  <Integer>^ - index of the second tree
}

See:  D.H. Huson and S. Linz. Autumn Algorithm�Computation of Hybridization Networks for Realistic Phylogenetic Trees. IEEE/ACM Transactions on Computational Biology and Bioinformatics: 15:398-420, 2018.

\subsubsection{Average Consensus}

The ``Average Consensus'' algorithm takes a Trees block as input and produces a Splits block as output. It calculates average consensus tree.

See: Francois-Joseph Lapointe, Guy Cucumel, The Average Consensus Procedure: Combination of Weighted Trees Containing Identical or Overlapping Sets of Taxa. Systematic Biology, 46(2):306-312 (1997).

\subsubsection{Average Distances}

The ``Average Distances'' algorithm takes a Trees block as input and produces a Distances block as output. It calculates the average distances between taxa over a set of trees.

See: Francois-Joseph Lapointe, Guy Cucumel, The Average Consensus Procedure: Combination of Weighted Trees Containing Identical or Overlapping Sets of Taxa. Systematic Biology, 46(2):306-312 (1997).

\subsubsection{Bootstrap Tree}

The ``Bootstrap Tree'' algorithm takes a Trees block as input and produces a Trees block as output. It performs bootstrapping on trees.

{\footnotesize\obeylines
\verb^Replicates =  <Integer>^ - number of bootstrap replicates
\verb^TransferBootstrap =  <Boolean>^ - use transform bootstrapping (TBE), less susceptible to rouge taxa
\verb^MinPercent =  <Double>^ - minimum percentage support for a branch to be included
\verb^RandomSeed =  <Integer>^ - if non-zero, is used as seed for random number generator
}

See: Felsenstein J. Confidence limits on phylogenies: an approach using the bootstrap. Evolution, 39(4):783-791 (1985);

\subsubsection{Bootstrap Tree Splits}

The ``Bootstrap Tree Splits'' algorithm takes a Trees block as input and produces a Splits block as output. It performs bootstrapping on trees.

{\footnotesize\obeylines
\verb^Replicates =  <Integer>^ - number of bootstrap replicates
\verb^MinPercent =  <Double>^ - minimum percentage support for a split to be included
\verb^ShowAllSplits =  <Boolean>^ - show all bootstrap splits, not just the original splits
\verb^RandomSeed =  <Integer>^ - if non-zero, is used as seed for random number generator
\verb^HighDimensionFilter =  <Boolean>^ - heuristically remove splits causing high-dimensional network
}

See: Felsenstein J. Confidence limits on phylogenies: an approach using the bootstrap. Evolution, 39(4):783-791 (1985);

\subsubsection{Cluster Network}

The ``Cluster Network'' algorithm takes a Trees block as input and produces a Trees block as output. It computes the cluster network that contains all input trees (in the hardwired sense).

{\footnotesize\obeylines
\verb^EdgeWeights =  {Mean | Count | Sum | Uniform}^ - compute edge weights
\verb^ThresholdPercent =  <Double>^ - minimum percentage of trees that a cluster must appear in
}

See:  D.H. Huson and R. Rupp (2008) Summarizing multiple gene trees using cluster networks. In: Crandall, K.A., Lagergren, J. (eds) Algorithms in Bioinformatics. WABI 2008. Lecture Notes in Computer Science(), vol 5251.

\subsubsection{Consensus Network}

The ``Consensus Network'' algorithm takes a Trees block as input and produces a Splits block as output. It computes the consensus network.

{\footnotesize\obeylines
\verb^EdgeWeights =  {Mean | TreeSizeWeightedMean | Median | Count | Sum | Uniform | TreeNormalizedSum}^ - how to calculate edge weights in resulting network
\verb^ThresholdPercent =  <Double>^ - threshold for percent of input trees that split has to occur in for it to appear in the output
\verb^HighDimensionFilter =  <Boolean>^ - heuristically remove splits causing high-dimensional consensus network
}

See:  B. Holland and V. Moulton. Consensus networks:  A method for visualizing incompatibilities in  collections  of  trees. In  G.  Benson  and  R.  Page,  editors, Proceedings  of  �Workshop  on Algorithms in Bioinformatics, volume 2812 of LNBI, pages 165�176. Springer, 2003.

\subsubsection{Consensus Outline}

The ``Consensus Outline'' algorithm takes a Trees block as input and produces a Splits block as output. It computes the consensus outline.

{\footnotesize\obeylines
\verb^EdgeWeights =  {Mean | TreeSizeWeightedMean | Median | Count | Sum | Uniform | TreeNormalizedSum}^ - how to calculate edge weights in resulting network
\verb^ThresholdPercent =  <Double>^ - threshold for percent of input trees that split has to occur in for it to appear in the output
}

See: DH Huson and B Cetinkaya, Visualizing incompatibilities in phylogenetic trees using consensus outlines, Front. Bioinform. (2023) 

\subsubsection{Consensus Splits}

The ``Consensus Splits'' algorithm takes a Trees block as input and produces a Splits block as output. It provides several consensus methods.

{\footnotesize\obeylines
\verb^Consensus =  {Strict | Majority | GreedyCompatible | ConsensusOutline | GreedyWeaklyCompatible | ConsensusNetwork}^ - consensus method
\verb^EdgeWeights =  {Mean | TreeSizeWeightedMean | Median | Count | Sum | Uniform | TreeNormalizedSum}^ - how to calculate edge weights in resulting network
\verb^ThresholdPercent =  <Double>^ - threshold for percent of input trees that split has to occur in for it to appear in the output
\verb^HighDimensionFilter =  <Boolean>^ - heuristically remove splits causing high-dimensional consensus network
}

See: D.H. Huson, R. Rupp and C. Scornavacca, Phylogenetic Networks, Cambridge, 2012.

\subsubsection{Consensus Tree}

The ``Consensus Tree'' algorithm takes a Trees block as input and produces a Trees block as output. It provides several methods for computing a consensus tree.

{\footnotesize\obeylines
\verb^Consensus =  {Majority | Greedy | Strict}^ - consensus method to use
\verb^EdgeWeights =  {Mean | TreeSizeWeightedMean | Median | Count | Sum | Uniform | TreeNormalizedSum}^ - determine how to calculate edge weights in resulting network
}

See: D. Bryant, A classification of consensus methods for phylogenetics, in Bioconsensus, 2001

\subsubsection{Enumerate Contained Trees}

The ``Enumerate Contained Trees'' algorithm takes a Trees block as input and produces a Trees block as output. It enumerates all contained trees.

{\footnotesize\obeylines
\verb^RemoveDuplicates =  <Boolean>^ - suppress duplicate trees in output
}

\subsubsection{Filtered Super Network}

The ``Filtered Super Network'' algorithm takes a Trees block as input and produces a Splits block as output. It computes a super network using the Z-closure method.

{\footnotesize\obeylines
\verb^MinNumberTrees =  <Integer>^ - set the min number trees
\verb^MaxDistortionScore =  <Integer>^ - set the max distortion score
\verb^UseTotalScore =  <Boolean>^ - set the use total score
}

See: James B. Whitfield, Sydney A. Cameron, Daniel H. Huson, Mike A. Steel. Filtered Z-Closure Supernetworks for Extracting and Visualizing Recurrent Signal from Incongruent Gene Trees, Systematic Biology, Volume 57, Issue 6, 1 December 2008, Pages 939�947.

\subsubsection{List One RSPR Trees}

The ``List One RSPR Trees'' algorithm takes a Trees block as input and produces a Report block as output. It determines which trees are exactly on rSPR away from each other.

{\footnotesize\obeylines
\verb^ApplyTo =  {OneTree | AllTrees}^ - determine whether to apply to one or all trees
\verb^WhichTree =  <Integer>^ - the index of the tree that the method will be applied to
}

\subsubsection{Loose And Lacy}

The ``Loose And Lacy'' algorithm takes a Trees block as input and produces a Trees block as output. It computes the `loose' and `lacy' species for a given tree and taxon trait.

{\footnotesize\obeylines
\verb^SpeciesDefinition =  {Loose | Lacy | Both}^ - species definition to use
\verb^TraitNumber =  <Integer>^ - number of specific trait to use
\verb^UseAllTraits =  <Boolean>^ - use all traits
}

See: Anica Hoppe, Sonja Tuerpitz, Mike Steel. Species notions that combine phylogenetic trees and phenotypic partitions. arXiv:1711.08145v1.

\subsubsection{Phylogenetic Diversity}

The ``Phylogenetic Diversity'' algorithm takes a Trees block as input and produces a Report block as output. It calculates the phylogenetic diversity for selected taxa.

{\footnotesize\obeylines
\verb^Rooted =  <Boolean>^ - interpret trees as rooted?
\verb^ApplyTo =  {OneTree | AllTrees}^ - determine whether to apply to one or all trees
\verb^WhichTree =  <Integer>^ - the index of the tree that the method will be applied to
}

See: Faith, D.P. Conservation evaluation and phylogenetic diversity. Biological Conservation 61, 1�10 (1992)

\subsubsection{Reroot Or Reorder Trees}

The ``Reroot Or Reorder Trees'' algorithm takes a Trees block as input and produces a Trees block as output. It reroot, or change the order of children, on all trees.

{\footnotesize\obeylines
\verb^RootBy =  {Off | MidPoint | OutGroup}^ - determine how to reroot
\verb^RearrangeBy =  {Off | RotateChildren | RotateSubTrees | ReverseChildren | ReverseSubTrees}^ - determine how to rearrange
\verb^Reorder =  {Off | ByTaxa | Lexicographically | ReverseOrder | LadderizedUp | LadderizedDown | LadderizedRandom | Stabilize}^ - determine how to reorder
}

\subsubsection{Rooted Consensus Tree}

The ``Rooted Consensus Tree'' algorithm takes a Trees block as input and produces a Trees block as output. It provides several methods for computing a rooted consensus tree.

{\footnotesize\obeylines
\verb^Consensus =  {Majority | Strict | Greedy}^ - consensus method to use
}

\subsubsection{Show Trees}

The ``Show Trees'' algorithm takes a Trees block as input and produces a View block as output. It provides several types of interactive visualizations of trees.

{\footnotesize\obeylines
\verb^View =  {TreeView | TreePages | Tanglegram | DensiTree}^ - the type of viewer to use
}

\subsubsection{Super Network}

The ``Super Network'' algorithm takes a Trees block as input and produces a Splits block as output. It computes a super network using the Z-closure method.

{\footnotesize\obeylines
\verb^EdgeWeights =  {AverageRelative | Mean | TreeSizeWeightedMean | Sum | Min | None}^ - determine how to calculate edge weights in resulting network
\verb^SuperTree =  <Boolean>^ - enforce the strong induction property, which results in a super tree
\verb^NumberOfRuns =  <Integer>^ - number of runs using random permutations of the input splits
\verb^ApplyRefineHeuristic =  <Boolean>^ - apply a simple refinement heuristic
\verb^Seed =  <Integer>^ - set seed used for random permutations
\verb^HighDimensionFilter =  <Boolean>^ - heuristically remove splits causing high-dimensional network
}

See: D.H. Huson, T. Dezulian, T. Kloepper, and M. A. Steel. Phylogenetic super-networks from partial trees. IEEE/ACM Transactions in Computational Biology and Bioinformatics, 1(4):151�158, 2004.

\subsubsection{Tree Diversity Index}

The ``Tree Diversity Index'' algorithm takes a Trees block as input and produces a Report block as output. It calculates the fair-proportion and equal-splits values on trees.

{\footnotesize\obeylines
\verb^Method =  {FairProportions | EqualSplits}^ - choose the type of index calculation
\verb^ApplyTo =  {OneTree | AllTrees}^ - determine whether to apply to one or all trees
\verb^WhichTree =  <Integer>^ - the index of the tree that the method will be applied to
}

See: Redding, D. Incorporating genetic distinctness and reserve occupancy into a conservation priorisation approach. Master�s thesis. University of East Anglia (2003)

\subsubsection{Tree Selector}

The ``Tree Selector'' algorithm takes a Trees block as input and produces a Trees block as output. It allows the user to select one from a list of trees.

{\footnotesize\obeylines
\verb^Which =  <Integer>^ - which tree to use
}

\subsubsection{Tree Selector Splits}

The ``Tree Selector Splits'' algorithm takes a Trees block as input and produces a Splits block as output. It selects a single tree and extracts its splits.

{\footnotesize\obeylines
    Which =  <Integer>
}

\subsubsection{Trees Filter}

The ``Trees Filter'' algorithm takes a Trees block as input and produces a Trees block as output. It allows the user to interactively filter trees.

\subsubsection{Trees Filter 2}

The ``Trees Filter 2'' algorithm takes a Trees block as input and produces a Trees block as output. It provides several options for filtering trees.

{\footnotesize\obeylines
\verb^RequireAllTaxa =  <Boolean>^ - keep only trees that have the full set of taxa
\verb^MinNumberOfTaxa =  <Integer>^ - keep only trees that have at least this number of taxa
\verb^MinNumberOfTaxa =  <Integer>^ - keep only trees that have at least this number of taxa
\verb^MinEdgeLength =  <Double>^ - keep only edges that have this minimum length
\verb^MinConfidence =  <Double>^ - keep only edges that have this minimum confidence value
\verb^UniformEdgeLengths =  <Boolean>^ - change all edge weights to 1
}

\subsubsection{Unrooted Shapley Values}

The ``Unrooted Shapley Values'' algorithm takes a Trees block as input and produces a Report block as output. It calculates unrooted Shapley values on trees.

{\footnotesize\obeylines
\verb^ApplyTo =  {OneTree | AllTrees}^ - determine whether to apply to one or all trees
\verb^WhichTree =  <Integer>^ - the index of the tree that the method will be applied to
}

See: Haake C.J., Kashiwada A., Su F.E. The Shapley value of phylogenetic trees. J Math Biol 56:479�497 (2008) 

\subsection{Input Network Block}

\subsubsection{Show Network}

The ``Show Network'' algorithm takes a Network block as input and produces a View block as output. It provides interactive visualizations of networks.

{\footnotesize\obeylines
\verb^View =  {Network | Text}^ - the type of network viewer to use
}

\end{document}